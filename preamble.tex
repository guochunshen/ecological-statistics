% 中文支持 - 使用 ctex 包提供更好的中文排版
\usepackage{ctex}

% 中文字体设置 - 提供多个选项,取消注释其中一个配置

% 选项1:思源宋体 + 思源黑体(现代、清晰的专业字体)
% \setCJKmainfont{Noto Serif CJK SC}[BoldFont={Noto Serif CJK SC Bold},ItalicFont={AR PL KaitiM GB}]
% \setCJKsansfont{Noto Sans CJK SC}
% \setCJKmonofont{Noto Sans Mono CJK SC}

% 选项2:文鼎字体系列(传统、优雅的中文字体)
%\setCJKmainfont{AR PL UMing CN}[BoldFont={AR PL UMing CN},ItalicFont={AR PL KaitiM GB}]
%\setCJKsansfont{AR PL SungtiL GB}
%\setCJKmonofont{Noto Sans Mono CJK SC}

% 选项3:系统默认字体(确保兼容性)
% \setCJKmainfont{}
% \setCJKsansfont{}
% \setCJKmonofont{}


% 英文字体设置
%\usepackage{fontspec}
%\setmainfont{DejaVu Serif}
%\setsansfont{DejaVu Sans}
%\setmonofont{DejaVu Sans Mono}

% 正文字体大小设置 - ctex 已处理中英文字体大小匹配

% 页面设置
\usepackage[top=2.5cm, bottom=2.5cm, left=3cm, right=3cm]{geometry}

% 行间距设置
\usepackage{setspace}
\onehalfspacing  % 1.5倍行间距

% 段落缩进设置
\usepackage{indentfirst}
\setlength{\parindent}{2em}  % 每段开头空两格

% 数学支持
\usepackage{amsmath,amssymb,amsthm}
\usepackage{bm}
\usepackage{booktabs}
\usepackage{longtable}
\makeatletter
\def\thm@space@setup{%
  \thm@preskip=8pt plus 2pt minus 4pt
  \thm@postskip=\thm@preskip
}
\makeatother


% 章节标题样式 - ctex 已提供优化的中文标题样式
% 如果需要自定义,可以取消注释以下设置:
% \usepackage{titlesec}
% \titleformat{\chapter}[display]
%   {\normalfont\Huge\bfseries\filcenter}{\chaptertitlename\ \thechapter}{20pt}{\Huge}
% \titleformat{\section}
%   {\normalfont\Large\bfseries}{\thesection}{1em}{}
% \titleformat{\subsection}
%   {\normalfont\large\bfseries}{\thesubsection}{1em}{}

% 图表标题 - ctex 已优化中文标题
\usepackage{caption}
\captionsetup{font=small,labelsep=quad}

% 超链接
\usepackage{hyperref}
\hypersetup{
  colorlinks=true,
  linkcolor=blue,
  filecolor=magenta,      
  urlcolor=cyan,
  pdftitle={生态统计学},
  pdfauthor={沈国春、李勤}
}

% 代码环境
\usepackage{listings}
\lstset{
  basicstyle=\footnotesize\ttfamily,  % 缩小代码字体
  breaklines=true,
  frame=single,
  numbers=left,
  numberstyle=\tiny,
  stepnumber=1,
  numbersep=5pt
}
