% 中文支持 - 使用 ctex 包提供更好的中文排版
\usepackage{ctex}

% 中文字体设置 - 使用 ctex 默认字体,不进行额外设置
% ctex 包会自动选择合适的中文字体


% 英文字体设置
%\usepackage{fontspec}
%\setmainfont{DejaVu Serif}
%\setsansfont{DejaVu Sans}
%\setmonofont{DejaVu Sans Mono}

% 正文字体大小设置 - ctex 已处理中英文字体大小匹配

% 页面设置
\usepackage[top=2cm, bottom=2cm, left=2.5cm, right=2.5cm]{geometry}

% 行间距设置
\usepackage{setspace}
\setstretch{1}  % 0.9倍行间距

% 段落缩进设置
\usepackage{indentfirst}
\setlength{\parindent}{2em}  % 每段开头空两格

% 数学支持
\usepackage{amsmath,amssymb,amsthm}
\usepackage{bm}
\usepackage{booktabs}
\usepackage{longtable}
\makeatletter
\def\thm@space@setup{%
  \thm@preskip=8pt plus 2pt minus 4pt
  \thm@postskip=\thm@preskip
}
\makeatother


% 章节标题样式 - ctex 已提供优化的中文标题样式
% 如果需要自定义,可以取消注释以下设置:
% \usepackage{titlesec}
% \titleformat{\chapter}[display]
%   {\normalfont\Huge\bfseries\filcenter}{\chaptertitlename\ \thechapter}{20pt}{\Huge}
% \titleformat{\section}
%   {\normalfont\Large\bfseries}{\thesection}{1em}{}
% \titleformat{\subsection}
%   {\normalfont\large\bfseries}{\thesubsection}{1em}{}

% 图表标题 - ctex 已优化中文标题
\usepackage{caption}
\captionsetup{font=small,labelsep=quad}

% 超链接 - 合并所有选项到单个配置
\usepackage[unicode]{hyperref}
\hypersetup{
  colorlinks=true,
  linkcolor=blue,
  filecolor=magenta,
  urlcolor=cyan,
  pdftitle={生态统计学AI+},
  pdfauthor={沈国春、李勤},
  bookmarks=true,
  bookmarksopen=true,
  bookmarksnumbered=true,
  pdfencoding=auto
}

% 解决重复标签问题
\usepackage[all]{hypcap}

% 防止重复标签警告
\usepackage{cleveref}
\crefname{figure}{图}{图}
\crefname{table}{表}{表}
\crefname{equation}{公式}{公式}

% 代码环境 - R代码块使用Verbatim和Shaded环境
\usepackage{listings}
\lstset{
  basicstyle=\footnotesize\ttfamily,
  breaklines=true,
  frame=single,
  numbers=left,
  numberstyle=\tiny,
  stepnumber=1,
  numbersep=5pt,
  lineskip=-1pt
}

% 调整R代码块(Shaded/Highlighting环境)的行间距
\usepackage{etoolbox}
\makeatletter
% 设置Verbatim环境(R代码块使用)的行间距
\AtBeginEnvironment{Shaded}{%
  \setstretch{0.9}%  % 代码块内行间距为0.8倍
}
% 设置代码块内空行的行高
\AtBeginEnvironment{Highlighting}{%
  \setlength{\baselineskip}{0.8\baselineskip}%  % 空行高度为0.8倍
}
% 设置verbatim环境(R代码输出)的行间距
\AtBeginEnvironment{verbatim}{%
  \setstretch{0.8}%  % verbatim环境行间距为0.8倍
  \setlength{\baselineskip}{0.8\baselineskip}%  % 空行高度为0.8倍
}
% 或者使用更精细的控制
%\preto\Shaded{\par\vspace{-0.5\baselineskip}}  % 代码块前减少空白
%\appto\endShaded{\par\vspace{-0.5\baselineskip}}  % 代码块后减少空白
\makeatother
