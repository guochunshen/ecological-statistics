% Options for packages loaded elsewhere
\PassOptionsToPackage{unicode}{hyperref}
\PassOptionsToPackage{hyphens}{url}
%
\documentclass[
]{book}
\usepackage{amsmath,amssymb}
\usepackage{iftex}
\ifPDFTeX
  \usepackage[T1]{fontenc}
  \usepackage[utf8]{inputenc}
  \usepackage{textcomp} % provide euro and other symbols
\else % if luatex or xetex
  \usepackage{unicode-math} % this also loads fontspec
  \defaultfontfeatures{Scale=MatchLowercase}
  \defaultfontfeatures[\rmfamily]{Ligatures=TeX,Scale=1}
\fi
\usepackage{lmodern}
\ifPDFTeX\else
  % xetex/luatex font selection
\fi
% Use upquote if available, for straight quotes in verbatim environments
\IfFileExists{upquote.sty}{\usepackage{upquote}}{}
\IfFileExists{microtype.sty}{% use microtype if available
  \usepackage[]{microtype}
  \UseMicrotypeSet[protrusion]{basicmath} % disable protrusion for tt fonts
}{}
\makeatletter
\@ifundefined{KOMAClassName}{% if non-KOMA class
  \IfFileExists{parskip.sty}{%
    \usepackage{parskip}
  }{% else
    \setlength{\parindent}{0pt}
    \setlength{\parskip}{6pt plus 2pt minus 1pt}}
}{% if KOMA class
  \KOMAoptions{parskip=half}}
\makeatother
\usepackage{xcolor}
\usepackage{color}
\usepackage{fancyvrb}
\newcommand{\VerbBar}{|}
\newcommand{\VERB}{\Verb[commandchars=\\\{\}]}
\DefineVerbatimEnvironment{Highlighting}{Verbatim}{commandchars=\\\{\}}
% Add ',fontsize=\small' for more characters per line
\usepackage{framed}
\definecolor{shadecolor}{RGB}{248,248,248}
\newenvironment{Shaded}{\begin{snugshade}}{\end{snugshade}}
\newcommand{\AlertTok}[1]{\textcolor[rgb]{0.94,0.16,0.16}{#1}}
\newcommand{\AnnotationTok}[1]{\textcolor[rgb]{0.56,0.35,0.01}{\textbf{\textit{#1}}}}
\newcommand{\AttributeTok}[1]{\textcolor[rgb]{0.13,0.29,0.53}{#1}}
\newcommand{\BaseNTok}[1]{\textcolor[rgb]{0.00,0.00,0.81}{#1}}
\newcommand{\BuiltInTok}[1]{#1}
\newcommand{\CharTok}[1]{\textcolor[rgb]{0.31,0.60,0.02}{#1}}
\newcommand{\CommentTok}[1]{\textcolor[rgb]{0.56,0.35,0.01}{\textit{#1}}}
\newcommand{\CommentVarTok}[1]{\textcolor[rgb]{0.56,0.35,0.01}{\textbf{\textit{#1}}}}
\newcommand{\ConstantTok}[1]{\textcolor[rgb]{0.56,0.35,0.01}{#1}}
\newcommand{\ControlFlowTok}[1]{\textcolor[rgb]{0.13,0.29,0.53}{\textbf{#1}}}
\newcommand{\DataTypeTok}[1]{\textcolor[rgb]{0.13,0.29,0.53}{#1}}
\newcommand{\DecValTok}[1]{\textcolor[rgb]{0.00,0.00,0.81}{#1}}
\newcommand{\DocumentationTok}[1]{\textcolor[rgb]{0.56,0.35,0.01}{\textbf{\textit{#1}}}}
\newcommand{\ErrorTok}[1]{\textcolor[rgb]{0.64,0.00,0.00}{\textbf{#1}}}
\newcommand{\ExtensionTok}[1]{#1}
\newcommand{\FloatTok}[1]{\textcolor[rgb]{0.00,0.00,0.81}{#1}}
\newcommand{\FunctionTok}[1]{\textcolor[rgb]{0.13,0.29,0.53}{\textbf{#1}}}
\newcommand{\ImportTok}[1]{#1}
\newcommand{\InformationTok}[1]{\textcolor[rgb]{0.56,0.35,0.01}{\textbf{\textit{#1}}}}
\newcommand{\KeywordTok}[1]{\textcolor[rgb]{0.13,0.29,0.53}{\textbf{#1}}}
\newcommand{\NormalTok}[1]{#1}
\newcommand{\OperatorTok}[1]{\textcolor[rgb]{0.81,0.36,0.00}{\textbf{#1}}}
\newcommand{\OtherTok}[1]{\textcolor[rgb]{0.56,0.35,0.01}{#1}}
\newcommand{\PreprocessorTok}[1]{\textcolor[rgb]{0.56,0.35,0.01}{\textit{#1}}}
\newcommand{\RegionMarkerTok}[1]{#1}
\newcommand{\SpecialCharTok}[1]{\textcolor[rgb]{0.81,0.36,0.00}{\textbf{#1}}}
\newcommand{\SpecialStringTok}[1]{\textcolor[rgb]{0.31,0.60,0.02}{#1}}
\newcommand{\StringTok}[1]{\textcolor[rgb]{0.31,0.60,0.02}{#1}}
\newcommand{\VariableTok}[1]{\textcolor[rgb]{0.00,0.00,0.00}{#1}}
\newcommand{\VerbatimStringTok}[1]{\textcolor[rgb]{0.31,0.60,0.02}{#1}}
\newcommand{\WarningTok}[1]{\textcolor[rgb]{0.56,0.35,0.01}{\textbf{\textit{#1}}}}
\usepackage{longtable,booktabs,array}
\usepackage{calc} % for calculating minipage widths
% Correct order of tables after \paragraph or \subparagraph
\usepackage{etoolbox}
\makeatletter
\patchcmd\longtable{\par}{\if@noskipsec\mbox{}\fi\par}{}{}
\makeatother
% Allow footnotes in longtable head/foot
\IfFileExists{footnotehyper.sty}{\usepackage{footnotehyper}}{\usepackage{footnote}}
\makesavenoteenv{longtable}
\usepackage{graphicx}
\makeatletter
\def\maxwidth{\ifdim\Gin@nat@width>\linewidth\linewidth\else\Gin@nat@width\fi}
\def\maxheight{\ifdim\Gin@nat@height>\textheight\textheight\else\Gin@nat@height\fi}
\makeatother
% Scale images if necessary, so that they will not overflow the page
% margins by default, and it is still possible to overwrite the defaults
% using explicit options in \includegraphics[width, height, ...]{}
\setkeys{Gin}{width=\maxwidth,height=\maxheight,keepaspectratio}
% Set default figure placement to htbp
\makeatletter
\def\fps@figure{htbp}
\makeatother
\setlength{\emergencystretch}{3em} % prevent overfull lines
\providecommand{\tightlist}{%
  \setlength{\itemsep}{0pt}\setlength{\parskip}{0pt}}
\setcounter{secnumdepth}{5}
% 中文支持
\usepackage{xeCJK}
\setCJKmainfont{Noto Serif CJK SC}
\setCJKsansfont{Noto Sans CJK SC} 
\setCJKmonofont{Noto Sans Mono CJK SC}

% 页面设置
\usepackage[top=2.5cm, bottom=2.5cm, left=3cm, right=3cm]{geometry}

% 数学支持
\usepackage{amsmath,amssymb,amsthm}
\usepackage{bm}

% 章节标题样式
\usepackage{titlesec}
\titleformat{\chapter}[display]
  {\normalfont\Huge\bfseries\filcenter}{\chaptertitlename\ \thechapter}{20pt}{\Huge}
\titleformat{\section}
  {\normalfont\Large\bfseries}{\thesection}{1em}{}
\titleformat{\subsection}
  {\normalfont\large\bfseries}{\thesubsection}{1em}{}

% 图表标题
\usepackage{caption}
\DeclareCaptionFont{heiti}{\CJKfamily{hei}}
\captionsetup{font=small,labelsep=quad}

% 超链接
\usepackage{hyperref}
\hypersetup{
  colorlinks=true,
  linkcolor=blue,
  filecolor=magenta,      
  urlcolor=cyan,
  pdftitle={生态统计学},
  pdfauthor={沈国春、李勤}
}

% 代码环境
\usepackage{listings}
\lstset{
  basicstyle=\small\ttfamily,
  breaklines=true,
  frame=single,
  numbers=left,
  numberstyle=\tiny,
  stepnumber=1,
  numbersep=5pt
}
\ifLuaTeX
  \usepackage{selnolig}  % disable illegal ligatures
\fi
\usepackage[]{natbib}
\bibliographystyle{apalike}
\IfFileExists{bookmark.sty}{\usepackage{bookmark}}{\usepackage{hyperref}}
\IfFileExists{xurl.sty}{\usepackage{xurl}}{} % add URL line breaks if available
\urlstyle{same}
\hypersetup{
  pdftitle={生态统计学},
  pdfauthor={沈国春、李勤},
  hidelinks,
  pdfcreator={LaTeX via pandoc}}

\title{生态统计学}
\author{沈国春、李勤}
\date{2025-08-08}

\begin{document}
\maketitle

{
\setcounter{tocdepth}{1}
\tableofcontents
}
\hypertarget{ux524dux8a00}{%
\chapter*{前言}\label{ux524dux8a00}}
\addcontentsline{toc}{chapter}{前言}

这是一本关于生态统计学的教材,旨在为生态学研究者提供实用的数据分析方法。本书结合R语言,介绍生态学研究中常用的统计技术。

本书特点:

\begin{itemize}
\tightlist
\item
  面向生态学研究实际问题
\item
  基于R语言实现
\item
  包含大量生态数据案例
\item
  循序渐进的教学安排
\end{itemize}

本书使用以下R包:

\begin{Shaded}
\begin{Highlighting}[]
\FunctionTok{install.packages}\NormalTok{(}\FunctionTok{c}\NormalTok{(}
  \StringTok{"tidyverse"}\NormalTok{, }\StringTok{"vegan"}\NormalTok{, }\StringTok{"lme4"}\NormalTok{, }\StringTok{"ggplot2"}\NormalTok{, }
  \StringTok{"bookdown"}\NormalTok{, }\StringTok{"knitr"}\NormalTok{, }\StringTok{"rmarkdown"}
\NormalTok{))}
\end{Highlighting}
\end{Shaded}

\hypertarget{intro}{%
\chapter{R语言入门准备}\label{intro}}

\hypertarget{ux5b66ux4e60ux76eeux6807}{%
\section{学习目标}\label{ux5b66ux4e60ux76eeux6807}}

\begin{itemize}
\tightlist
\item
  理解为什么生态学专业需要学习R语言
\item
  掌握R和RStudio的安装和配置
\item
  建立良好的项目文件组织习惯
\item
  了解R包管理的基本方法
\end{itemize}

\hypertarget{ux4e3aux4ec0ux4e48ux751fux6001ux5b66ux4e13ux4e1aux9700ux8981ux5b66ux4e60r}{%
\section{为什么生态学专业需要学习R?}\label{ux4e3aux4ec0ux4e48ux751fux6001ux5b66ux4e13ux4e1aux9700ux8981ux5b66ux4e60r}}

\hypertarget{ux6570ux636eux5206ux6790ux80fdux529bux7684ux5fc5ux8981ux6027}{%
\subsection{数据分析能力的必要性}\label{ux6570ux636eux5206ux6790ux80fdux529bux7684ux5fc5ux8981ux6027}}

现代生态学研究产生海量数据:野外调查数据、实验室测量数据、遥感影像数据、基因序列数据等。传统的Excel已无法满足复杂的统计分析需求,而R语言提供了完整的数据科学工具链。

\begin{Shaded}
\begin{Highlighting}[]
\CommentTok{\# 示例:读取大型CSV文件}
\NormalTok{data }\OtherTok{\textless{}{-}} \FunctionTok{read.csv}\NormalTok{(}\StringTok{"large\_ecological\_data.csv"}\NormalTok{, }\AttributeTok{nrows=}\DecValTok{100000}\NormalTok{)}
\end{Highlighting}
\end{Shaded}

\hypertarget{ux53efux91cdux73b0ux7814ux7a76ux7684ux79d1ux5b66ux8981ux6c42}{%
\subsection{可重现研究的科学要求}\label{ux53efux91cdux73b0ux7814ux7a76ux7684ux79d1ux5b66ux8981ux6c42}}

\begin{itemize}
\tightlist
\item
  \textbf{重现性危机}:越来越多的科学研究无法被重现,影响科学可信度
\item
  \textbf{R脚本的优势}:每一步分析都有记录,任何人都可以重现你的分析过程
\end{itemize}

\begin{Shaded}
\begin{Highlighting}[]
\CommentTok{\# 示例:可重现的分析流程}
\CommentTok{\# 1. 数据导入}
\NormalTok{data }\OtherTok{\textless{}{-}} \FunctionTok{read.csv}\NormalTok{(}\StringTok{"field\_data.csv"}\NormalTok{)}

\CommentTok{\# 2. 数据清洗}
\NormalTok{clean\_data }\OtherTok{\textless{}{-}} \FunctionTok{na.omit}\NormalTok{(data)}

\CommentTok{\# 3. 统计分析}
\NormalTok{result }\OtherTok{\textless{}{-}} \FunctionTok{lm}\NormalTok{(response }\SpecialCharTok{\textasciitilde{}}\NormalTok{ predictor, }\AttributeTok{data=}\NormalTok{clean\_data)}
\end{Highlighting}
\end{Shaded}

\hypertarget{ux73afux5883ux8bbeux7f6eux548cux914dux7f6e}{%
\subsection{环境设置和配置}\label{ux73afux5883ux8bbeux7f6eux548cux914dux7f6e}}

\hypertarget{rux548crstudioux5b89ux88c5}{%
\subsubsection{R和RStudio安装}\label{rux548crstudioux5b89ux88c5}}

\begin{Shaded}
\begin{Highlighting}[]
\CommentTok{\# 检查R版本}
\NormalTok{R.version.string}

\CommentTok{\# 检查已安装的包}
\FunctionTok{installed.packages}\NormalTok{()}
\end{Highlighting}
\end{Shaded}

\hypertarget{ux9879ux76eeux6587ux4ef6ux5939ux7ed3ux6784}{%
\subsubsection{项目文件夹结构}\label{ux9879ux76eeux6587ux4ef6ux5939ux7ed3ux6784}}

\begin{Shaded}
\begin{Highlighting}[]
\CommentTok{\# 创建项目目录结构示例}
\FunctionTok{dir.create}\NormalTok{(}\StringTok{"生态学R语言课程"}\NormalTok{)}
\FunctionTok{dir.create}\NormalTok{(}\StringTok{"生态学R语言课程/数据"}\NormalTok{)}
\FunctionTok{dir.create}\NormalTok{(}\StringTok{"生态学R语言课程/脚本"}\NormalTok{)}
\FunctionTok{dir.create}\NormalTok{(}\StringTok{"生态学R语言课程/结果"}\NormalTok{)}
\end{Highlighting}
\end{Shaded}

\hypertarget{ux5305ux7ba1ux7406}{%
\subsubsection{包管理}\label{ux5305ux7ba1ux7406}}

\begin{Shaded}
\begin{Highlighting}[]
\CommentTok{\# 安装生态学常用包}
\FunctionTok{install.packages}\NormalTok{(}\FunctionTok{c}\NormalTok{(}\StringTok{"vegan"}\NormalTok{, }\StringTok{"lme4"}\NormalTok{, }\StringTok{"ggplot2"}\NormalTok{, }\StringTok{"dplyr"}\NormalTok{))}

\CommentTok{\# 加载包}
\FunctionTok{library}\NormalTok{(vegan)}
\FunctionTok{library}\NormalTok{(ggplot2)}
\end{Highlighting}
\end{Shaded}

\hypertarget{rux8bedux8a00ux7b80ux4ecbux4e0eux8da3ux5473ux5e94ux7528}{%
\section{R语言简介与趣味应用}\label{rux8bedux8a00ux7b80ux4ecbux4e0eux8da3ux5473ux5e94ux7528}}

\hypertarget{ux4ec0ux4e48ux662frux8bedux8a00}{%
\subsection{什么是R语言?}\label{ux4ec0ux4e48ux662frux8bedux8a00}}

\begin{itemize}
\tightlist
\item
  \textbf{统计计算语言}:专门为统计分析和数据可视化设计的编程语言
\item
  \textbf{开源免费}:由全球统计学家共同维护发展
\item
  \textbf{扩展性强}:超过18,000个专业扩展包
\end{itemize}

\hypertarget{ux8da3ux5473ux5e94ux7528ux793aux4f8b}{%
\subsection{趣味应用示例}\label{ux8da3ux5473ux5e94ux7528ux793aux4f8b}}

\hypertarget{ux751fux6001ux6570ux636eux52a8ux6001ux53efux89c6ux5316}{%
\subsubsection{生态数据动态可视化}\label{ux751fux6001ux6570ux636eux52a8ux6001ux53efux89c6ux5316}}

\begin{Shaded}
\begin{Highlighting}[]
\CommentTok{\# 安装必要包}
\FunctionTok{install.packages}\NormalTok{(}\FunctionTok{c}\NormalTok{(}\StringTok{"ggplot2"}\NormalTok{, }\StringTok{"gganimate"}\NormalTok{, }\StringTok{"gapminder"}\NormalTok{))}

\CommentTok{\# 绘制动态变化图}
\FunctionTok{library}\NormalTok{(ggplot2)}
\FunctionTok{library}\NormalTok{(gganimate)}
\FunctionTok{ggplot}\NormalTok{(gapminder, }\FunctionTok{aes}\NormalTok{(gdpPercap, lifeExp, }\AttributeTok{size =}\NormalTok{ pop, }\AttributeTok{color =}\NormalTok{ continent)) }\SpecialCharTok{+}
  \FunctionTok{geom\_point}\NormalTok{() }\SpecialCharTok{+}
  \FunctionTok{scale\_size}\NormalTok{(}\AttributeTok{range =} \FunctionTok{c}\NormalTok{(}\DecValTok{2}\NormalTok{, }\DecValTok{12}\NormalTok{)) }\SpecialCharTok{+}
  \FunctionTok{scale\_x\_log10}\NormalTok{() }\SpecialCharTok{+}
  \FunctionTok{labs}\NormalTok{(}\AttributeTok{title =} \StringTok{\textquotesingle{}年份: \{frame\_time\}\textquotesingle{}}\NormalTok{, }\AttributeTok{x =} \StringTok{\textquotesingle{}人均GDP\textquotesingle{}}\NormalTok{, }\AttributeTok{y =} \StringTok{\textquotesingle{}预期寿命\textquotesingle{}}\NormalTok{) }\SpecialCharTok{+}
  \FunctionTok{transition\_time}\NormalTok{(year) }\SpecialCharTok{+}
  \FunctionTok{ease\_aes}\NormalTok{(}\StringTok{\textquotesingle{}linear\textquotesingle{}}\NormalTok{)}
\end{Highlighting}
\end{Shaded}

\hypertarget{ux751fux6210ux97f3ux4e50}{%
\subsubsection{生成音乐}\label{ux751fux6210ux97f3ux4e50}}

\begin{Shaded}
\begin{Highlighting}[]
\FunctionTok{install.packages}\NormalTok{(}\StringTok{"audio"}\NormalTok{)}
\FunctionTok{library}\NormalTok{(audio)}
\FunctionTok{play}\NormalTok{(}\FunctionTok{sin}\NormalTok{(}\DecValTok{1}\SpecialCharTok{:}\DecValTok{10000}\SpecialCharTok{/}\DecValTok{10}\NormalTok{))}
\end{Highlighting}
\end{Shaded}

\hypertarget{ux5982ux4f55ux83b7ux53d6ux5e2eux52a9}{%
\subsection{如何获取帮助?}\label{ux5982ux4f55ux83b7ux53d6ux5e2eux52a9}}

\begin{Shaded}
\begin{Highlighting}[]
\CommentTok{\# 查看函数帮助文档}
\NormalTok{?plot}
\FunctionTok{help}\NormalTok{(}\StringTok{"plot"}\NormalTok{)}

\CommentTok{\# 搜索帮助文档}
\NormalTok{??}\StringTok{"regression"}

\CommentTok{\# 在线资源:}
\CommentTok{\# {-} R官方文档:https://cran.r{-}project.org/manuals.html}
\CommentTok{\# {-} RStudio社区:https://community.rstudio.com}
\end{Highlighting}
\end{Shaded}

\hypertarget{ux8bfeux524dux51c6ux5907ux68c0ux67e5ux6e05ux5355}{%
\section{课前准备检查清单}\label{ux8bfeux524dux51c6ux5907ux68c0ux67e5ux6e05ux5355}}

\begin{itemize}
\tightlist
\item[$\square$]
  R软件安装成功
\item[$\square$]
  RStudio安装成功\\
\item[$\square$]
  能够运行简单的R代码
\item[$\square$]
  创建了课程文件夹结构
\item[$\square$]
  理解了学习R语言的重要性
\item[$\square$]
  尝试运行了趣味示例代码
\end{itemize}

\hypertarget{forest-survey}{%
\chapter{森林调查数据计算}\label{forest-survey}}

\hypertarget{ux751fux6001ux5b66ux80ccux666f}{%
\section{生态学背景}\label{ux751fux6001ux5b66ux80ccux666f}}

在野外森林调查中,测量树木胸径(DBH,距地面1.3米处的直径)是评估森林生长状况的基本方法。我们需要计算样地内树木的平均胸径来了解林分特征。

\hypertarget{ux6f14ux793aux6570ux636e}{%
\section{演示数据}\label{ux6f14ux793aux6570ux636e}}

\begin{Shaded}
\begin{Highlighting}[]
\CommentTok{\# 某样地内10棵马尾松的胸径测量值(单位:厘米)}
\NormalTok{tree\_dbh }\OtherTok{\textless{}{-}} \FunctionTok{c}\NormalTok{(}\FloatTok{15.2}\NormalTok{, }\FloatTok{18.7}\NormalTok{, }\FloatTok{22.1}\NormalTok{, }\FloatTok{19.5}\NormalTok{, }\FloatTok{16.8}\NormalTok{, }\FloatTok{20.3}\NormalTok{, }\FloatTok{17.9}\NormalTok{, }\FloatTok{21.4}\NormalTok{, }\FloatTok{19.2}\NormalTok{, }\FloatTok{18.6}\NormalTok{)}
\end{Highlighting}
\end{Shaded}

\hypertarget{ux8bfeux5802ux6f14ux793aux8fc7ux7a0b}{%
\section{课堂演示过程}\label{ux8bfeux5802ux6f14ux793aux8fc7ux7a0b}}

\begin{Shaded}
\begin{Highlighting}[]
\CommentTok{\# 1. 创建胸径数据向量}
\NormalTok{tree\_dbh }\OtherTok{\textless{}{-}} \FunctionTok{c}\NormalTok{(}\FloatTok{15.2}\NormalTok{, }\FloatTok{18.7}\NormalTok{, }\FloatTok{22.1}\NormalTok{, }\FloatTok{19.5}\NormalTok{, }\FloatTok{16.8}\NormalTok{, }\FloatTok{20.3}\NormalTok{, }\FloatTok{17.9}\NormalTok{, }\FloatTok{21.4}\NormalTok{, }\FloatTok{19.2}\NormalTok{, }\FloatTok{18.6}\NormalTok{)}

\CommentTok{\# 2. 查看数据}
\NormalTok{tree\_dbh}
\end{Highlighting}
\end{Shaded}

\begin{verbatim}
##  [1] 15.2 18.7 22.1 19.5 16.8 20.3 17.9 21.4 19.2 18.6
\end{verbatim}

\begin{Shaded}
\begin{Highlighting}[]
\FunctionTok{length}\NormalTok{(tree\_dbh)  }\CommentTok{\# 查看测量了多少棵树}
\end{Highlighting}
\end{Shaded}

\begin{verbatim}
## [1] 10
\end{verbatim}

\begin{Shaded}
\begin{Highlighting}[]
\CommentTok{\# 3. 计算平均胸径}
\NormalTok{mean\_dbh }\OtherTok{\textless{}{-}} \FunctionTok{mean}\NormalTok{(tree\_dbh)}
\NormalTok{mean\_dbh}
\end{Highlighting}
\end{Shaded}

\begin{verbatim}
## [1] 18.97
\end{verbatim}

\begin{Shaded}
\begin{Highlighting}[]
\CommentTok{\# 4. 计算总树数}
\NormalTok{tree\_count }\OtherTok{\textless{}{-}} \FunctionTok{length}\NormalTok{(tree\_dbh)}
\NormalTok{tree\_count}
\end{Highlighting}
\end{Shaded}

\begin{verbatim}
## [1] 10
\end{verbatim}

\begin{Shaded}
\begin{Highlighting}[]
\CommentTok{\# 5. 简单的数学运算}
\NormalTok{max\_dbh }\OtherTok{\textless{}{-}} \FunctionTok{max}\NormalTok{(tree\_dbh)  }\CommentTok{\# 最大胸径}
\NormalTok{min\_dbh }\OtherTok{\textless{}{-}} \FunctionTok{min}\NormalTok{(tree\_dbh)  }\CommentTok{\# 最小胸径}
\NormalTok{total\_dbh }\OtherTok{\textless{}{-}} \FunctionTok{sum}\NormalTok{(tree\_dbh)  }\CommentTok{\# 胸径总和}
\end{Highlighting}
\end{Shaded}

\hypertarget{rux8bedux8a00ux77e5ux8bc6ux70b9ux8be6ux89e3}{%
\section{R语言知识点详解}\label{rux8bedux8a00ux77e5ux8bc6ux70b9ux8be6ux89e3}}

\hypertarget{ux5411ux91cfux521bux5efaux51fdux6570-c}{%
\subsection{\texorpdfstring{向量创建函数 \texttt{c()}}{向量创建函数 c()}}\label{ux5411ux91cfux521bux5efaux51fdux6570-c}}

\begin{itemize}
\tightlist
\item
  \textbf{是什么}:\texttt{c()} 是combine的缩写,用于将多个值组合成一个向量
\item
  \textbf{语法}:\texttt{c(值1,\ 值2,\ 值3,\ ...)}
\item
  \textbf{重要特点}:

  \begin{itemize}
  \tightlist
  \item
    向量中的所有元素必须是同一类型(数值、字符或逻辑)
  \item
    如果混合不同类型,R会自动转换为最通用的类型
  \item
    向量是R中最基本的数据结构
  \end{itemize}
\end{itemize}

\hypertarget{ux57faux672cux7edfux8ba1ux51fdux6570}{%
\subsection{基本统计函数}\label{ux57faux672cux7edfux8ba1ux51fdux6570}}

\hypertarget{mean---ux8ba1ux7b97ux5e73ux5747ux503c}{%
\subsubsection{\texorpdfstring{\texttt{mean()} - 计算平均值}{mean() - 计算平均值}}\label{mean---ux8ba1ux7b97ux5e73ux5747ux503c}}

\begin{Shaded}
\begin{Highlighting}[]
\CommentTok{\# 示例}
\FunctionTok{mean}\NormalTok{(}\FunctionTok{c}\NormalTok{(}\DecValTok{1}\NormalTok{, }\DecValTok{2}\NormalTok{, }\DecValTok{3}\NormalTok{, }\DecValTok{4}\NormalTok{, }\DecValTok{5}\NormalTok{))}
\end{Highlighting}
\end{Shaded}

\begin{verbatim}
## [1] 3
\end{verbatim}

\hypertarget{length---ux8ba1ux7b97ux5411ux91cfux957fux5ea6}{%
\subsubsection{\texorpdfstring{\texttt{length()} - 计算向量长度}{length() - 计算向量长度}}\label{length---ux8ba1ux7b97ux5411ux91cfux957fux5ea6}}

\begin{Shaded}
\begin{Highlighting}[]
\CommentTok{\# 示例}
\FunctionTok{length}\NormalTok{(}\FunctionTok{c}\NormalTok{(}\DecValTok{1}\NormalTok{, }\DecValTok{2}\NormalTok{, }\DecValTok{3}\NormalTok{))}
\end{Highlighting}
\end{Shaded}

\begin{verbatim}
## [1] 3
\end{verbatim}

\hypertarget{ux8bfeux540eux7ec3ux4e60}{%
\section{课后练习}\label{ux8bfeux540eux7ec3ux4e60}}

\textbf{题目}:某湿地调查中测量了8棵柳树的树高(单位:米):
\texttt{tree\_height\ \textless{}-\ c(4.2,\ 5.1,\ 3.8,\ 4.7,\ 5.3,\ 4.9,\ 4.1,\ 4.6)}

请完成:
1. 创建树高向量并查看数据
2. 计算平均树高
3. 找出最高的树有多高
4. 统计总共测量了多少棵树
5. 计算所有树的总高度
6. 计算树高的标准差(提示:使用sd()函数)

\begin{Shaded}
\begin{Highlighting}[]
\CommentTok{\# 在此处编写你的练习代码}
\end{Highlighting}
\end{Shaded}


  \bibliography{book.bib,packages.bib}

\end{document}
